
\documentclass[12pt, twoside]{report}
\begin{document}
\relax 
\providecommand\hyper@newdestlabel[2]{
La industria aeronáutica ha sufrido, en los últimos veinte años, un cambio profundo en su contexto industrial. Debido a la entrada en el mercado civil de nuevos actores y unido a la reducción del gasto militar que acompañó el final de la guerra fría, la eficiencia y la competitividad han ganado importancia para la permanencia en el sector. Los sistemas de producción lean y la estandarización se han utilizado masivamente como método de mejora continua. Dentro de este contexto y debido a su impacto en todos los ámbitos de la empresa, la mejora de la planificación a todos los niveles es un facilitador fundamental. Por tanto, ser capaz de producir una planificación de detalle factible y exacta se ha convertido en prioritario.

Además, el secuenciado de tareas con recursos limitados es un problema de optimización NP-complejo. Más aún, es uno de los problemas más difíciles de tratar. Tanto por su relevancia industrial como por su dificultad técnica, la resolución de problemas de secuenciado con recursos limitados está siendo objeto de numerosas investigaciones. A pesar de esto, los problemas de secuenciado que suelen ser estudiados en la literatura no cubren todas las características de muchos problemas reales, entre los que se encuentra el secuenciado de tareas dentro de las plataformas de montaje aeronáuticas.

En este trabajo se ha utilizado el programa C para resolver un problema de secuenciado de actividades multimodo en un entorno de tiempo limitado.
Se trata de formulaciones de aplicabilidad directa a casos reales de diferentes
industrias, entre las que se encuentra la industria aeronáutica.}
\@setckpt{Pages/Resumen}{
\setcounter{page}{1}
\setcounter{equation}{0}
\setcounter{enumi}{0}
\setcounter{enumii}{0}
\setcounter{enumiii}{0}
\setcounter{enumiv}{0}
\setcounter{footnote}{0}
\setcounter{mpfootnote}{0}
\setcounter{part}{0}
\setcounter{chapter}{0}
\setcounter{section}{0}
\setcounter{subsection}{0}
\setcounter{subsubsection}{0}
\setcounter{paragraph}{0}
\setcounter{subparagraph}{0}
\setcounter{figure}{0}
\setcounter{table}{0}
\setcounter{lstnumber}{1}
\setcounter{nlinenum}{0}
\setcounter{ContinuedFloat}{0}
\setcounter{float@type}{8}
\setcounter{parentequation}{0}
\setcounter{@pps}{0}
\setcounter{@ppsavesec}{0}
\setcounter{@ppsaveapp}{0}
\setcounter{@index}{0}
\setcounter{@plane}{0}
\setcounter{@row}{0}
\setcounter{@col}{0}
\setcounter{use@args}{0}
\setcounter{@record}{0}
\setcounter{arg@index}{0}
\setcounter{break@count}{0}
\setcounter{index@count}{0}
\setcounter{loop@count}{0}
\setcounter{VerbboxLineNo}{0}
\setcounter{Item}{0}
\setcounter{Hfootnote}{0}
\setcounter{bookmark@seq@number}{0}
\setcounter{lips@count}{0}
\setcounter{lstlisting}{0}
\setcounter{section@level}{0}
}



\end{document}